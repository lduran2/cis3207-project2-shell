\documentclass{article}
\usepackage[utf8]{inputenc}

\usepackage{amsmath}
\title{CIS 3207\\*Project 2A.\\*Shell Pseudocode}
\author{Leomar Dur\'an}
\date{\(19^{\text{th}}\) February 2020}

\usepackage[bookmarksopen=true]{hyperref}
\usepackage{bookmark}

\usepackage{enumitem}
\setlistdepth{9}
\newlist{deepenum}{enumerate}{10}
\setlist[deepenum,1]{label=\arabic*.}
\setlist[deepenum,2]{label=(\alph*)}
\setlist[deepenum,3]{label=(\roman*)}
\setlist[deepenum,4]{label=\Alph*.}
\setlist[deepenum,5]{label=\Roman*.}
\setlist[deepenum,6]{label=\arabic*.}
\setlist[deepenum,7]{label=(\alph*)}
\setlist[deepenum,8]{label=(\roman*)}
\setlist[deepenum,9]{label=\Alph*.}
\setlist[deepenum,10]{label=\Roman*.}

\begin{document}

\maketitle

\section{Main program}

\begin{deepenum}
    \item
        Start a loop while \textbf{Calling} prompting.
        \begin{deepenum}
            \item
                \textbf{Call} Parse the commandline string.
            \item
                Set up the redirects for file input and output.
            \item
                Depending on the first argument in the parsing:
                \begin{deepenum}
                    \item
                        Case ``\texttt{cd}": \textbf{Call} Change the directory.
                    \item
                        Case ``\texttt{clr}": \textbf{Call} Clear the screen.
                    \item
                        Case ``\texttt{dir}": \textbf{Call} Print the directory listing. See 2020--02--19 (Shell, Week 2) (PPTX), slide 8 for the implementation.
                    \item
                        Case ``\texttt{env}": \textbf{Call} Access the environment variable.
                    \item
                        Case ``\texttt{echo}": \textbf{Call} Echo the commandline.
                    \item
                        Case ``\texttt{help}": \textbf{Call} Ask help with the current command.
                    \item
                        Case ``\texttt{pause}": \textbf{Call} Pause until the user hits Enter.
                    \item
                        Default: \textbf{Call} Run an external command.
                \end{deepenum}
            \item
                If parallel next, set up the next file to run concurrent to this one.
            \item
                If pipe out, then wait before running the next file.
        \end{deepenum}
    \item
        Loop back.
\end{deepenum}

\section{Prompt}
\begin{deepenum}
    \item
        Check the prompt environment variable.
    \item
        Decode it and print it as appropriate.
    \item
        Accept a commandline string from the prompt.
    \item
        If the string contains an end-of-file character (Ctrl+\texttt{D}), is the string ``\texttt{exit}", or the string ``\texttt{quit}", return false.
    \item
        Return true and the commandline string.
\end{deepenum}

\section{Process structure}
\begin{itemize}
    \item
        command name : string
    \item
        arguments : string[]
    \item
        input file : FILE*
    \item
        input append : Boolean
    \item
        output file : FILE*
    \item
        pipe in : Process*
    \item
        pipe out : Process*
    \item
        parallel next : Process*
\end{itemize}

\section{Parse the commandline string}
\begin{deepenum}
    \item
        Let \(\text{redirect characters} := \{``<", ``>", ``|", ``\&" \}\).
    \item
        Create a queue of strings.
    \item
        Initialize a left counter to zero.
    \item
        Initialize a tokenizer state that can be ``in text", ``in space", ``in redirect" to ``in space".
    \item
        Initialize a in-quote state that can be ``outside quotes", ``in double quotes" or ``in single quotes" to ``outside quotes".
    \item
        Start a loop until the null character `\texttt{\textbackslash0}' is reached.
        \begin{deepenum}
            \item
                Depending on the in-quote state:
                \begin{deepenum}
                    \item
                        Case ``outside quotes":
                        \begin{deepenum}
                            \item
                                If the current character is a single quote:
                                \begin{deepenum}
                                    \item
                                        Change the state to ``in single quotes".
                                \end{deepenum}
                            \item
                                Otherwise, if the current character is in double quotes:
                                \begin{deepenum}
                                    \item
                                        Change the state to ``in double quotes".
                                \end{deepenum}
                        \end{deepenum}
                    \item
                        Case ``in single quotes":
                        \begin{deepenum}
                            \item
                                If the current character is a single quote:
                                    \begin{deepenum}
                                        \item
                                            Change the state to ``outside quotes".
                                    \end{deepenum}
                            \item
                                Otherwise, skip this iteration of the loop.
                        \end{deepenum}
                    \item
                        Case ``in double quotes":
                        \begin{deepenum}
                            \item
                                If the current character is a double quote:
                                    \begin{deepenum}
                                        \item
                                            Change the state to ``outside quotes".
                                    \end{deepenum}
                            \item
                                Otherwise, skip this iteration of the loop.
                        \end{deepenum}
                \end{deepenum}
            \item
                Depending on the tokenizer state:
                \begin{deepenum}
                    \item
                        Case ``in text":
                            \begin{deepenum}
                                \item
                                    If the current character is a space:
                                    \begin{deepenum}
                                        \item
                                            Change the state to ``in space".
                                        \item
                                            Copy a substring from the left counter to one less than the current character to the.
                                        \item
                                            Enqueue the substring.
                                    \end{deepenum}
                                \item
                                    If the current character is in the list of redirect characters:
                                    \begin{deepenum}
                                        \item
                                            Remember the character.
                                        \item
                                            Change the state to ``in redirect".
                                    \end{deepenum}
                            \end{deepenum}
                    \item
                        Case ``in space":
                            \begin{deepenum}
                                \item
                                    If the current character is a non-space character:
                                        \begin{deepenum}
                                            \item
                                                Change the state to ``in text".
                                            \item
                                                Set the left counter to the current character.
                                            \item
                                                Stay on this character.
                                        \end{deepenum}
                            \end{deepenum}
                    \item
                        Case ``in redirect":
                            \begin{deepenum}
                                \item
                                    If the current character is another redirect character:
                                        \begin{deepenum}
                                            \item
                                                Copy the characters from the original redirect to this character to redirect.
                                            \item
                                                Depending on the original character, place this as an input file, output file, pipe out, or parallel next of the current file.
                                            \item
                                                If this is a pipe out, mark the current file as its pipe in, and change to this file.
                                            \item
                                                If this is parallel next, change to this file.
                                        \end{deepenum}
                            \end{deepenum}
                \end{deepenum}
        \end{deepenum}
    \item
        Loop back.
    \item
        Create an array with the same size as the queue.
    \item
        While the queue has a next string.
        \begin{deepenum}
            \item
                Dequeue the item into the array.
        \end{deepenum}
    \item
        Return the length and the starting position of the array.
\end{deepenum}

\section{Change the directory}
\begin{deepenum}
    \item
        Use \texttt{chdir} on the second argument.
    \item
        If the result is nonzero, check \texttt{errno} and print the directory name and error message.
\end{deepenum}

\section{Clear the screen}
\begin{deepenum}
    \item
        Print the escape sequence ``\verb|\033[H\033[2J|".
\end{deepenum}

\section{Access the environment variable.}
\begin{deepenum}
    \item
        Depending on the number of arguments:
        \begin{deepenum}
            \item
                Case 1:
                \begin{deepenum}
                    \item
                        Loop through the environment variables and print them one per line.
                \end{deepenum}
            \item
                Case 2:
                \begin{deepenum}
                    \item
                        Print the value of the environment variable specified by the second argument.
                \end{deepenum}
            \item
                Case 3:
                \begin{deepenum}
                    \item
                        Replace the value of the environment variable specified by the second argument by the third argument.
                \end{deepenum}
        \end{deepenum}
\end{deepenum}

\section{Echo the commandline}
\begin{deepenum}
    \item
        Search for the first instead of ``echo" in the original prompt.
    \item
        Print every character one character after that until any other special characters, and a newline.
\end{deepenum}

\section{Ask help with the current command}
\begin{deepenum}
    \item
        Check if the second argument is an internal command.
        \begin{deepenum}
            \item
                If so, print a manual page for that command.
            \item
                If not, \textbf{Call} Run the external help command.
        \end{deepenum}
    \item
        Use the \texttt{more} filter to split up the pages.
\end{deepenum}

\section{Pause until the user hits [Enter]}
\begin{deepenum}
    \item
        Print ``Please press [Enter] to continue . . ."
    \item
        Loop until the user hits enter.
        \begin{deepenum}
            \item Get a character from the user.
            \item Flush the input stream.
            \item Check if the character was newline or carriage return.
        \end{deepenum}
\end{deepenum}

\section{Run an external command}
\begin{deepenum}
    \item
        If the command is not help:
        \begin{deepenum}
            \item
                Check if it exists in the current directory.
            \item
                If it does not exist in the current directory:
                \begin{deepenum}
                    \item
                        Loop through the paths stored in the \texttt{\$PATH} variable.
                    \item
                        Check each path for a command with that name until it is found.
                \end{deepenum}
        \end{deepenum}
    \item
        Create a child process.
    \item
        Launch the external command using the new child process in its appropriate directory.
\end{deepenum}

\end{document}

